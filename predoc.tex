%! Author = aazah
%! Date = 16.12.2025
%
\documentclass[10pt,a4paper]{moderncv}
%\renewcommand*{\addresssymbol}{\CircledA}
\moderncvtheme[cerulean]{contemporary}
\usepackage{fontspec}
%\setmainfont{CMU Serif}[Ligatures=TeX]
\usepackage[russian, english]{babel}
\usepackage[T1]{fontenc}
%\usepackage[T2A]{fontenc}
%\usepackage[russian, english]{babel} % Add 'russian' to the babel package options
%Prior to 2018, LaTeX’s handling of input files encoded in UTF-8 required users to add the line
%
%\usepackage[utf8]{inputenc}
%Prior to 2018, LaTeX’s handling of input files encoded in UTF-8 required users to add the line to their document preamble.
%With the release of TeX Live 2018, LaTeX was enhanced to adopt UTF-8 as its default text encoding,
%removing the need to add \usepackage[utf8]{inputenc}
%\usepackage[utf8]{inputenc}
\definecolor{addresscolor}{rgb}{0.35,0.35,0.35}
%\usepackage{lmodern}
\usepackage[hmargin=0.5in,vmargin=10pt]{geometry}

\usepackage{url}
\usepackage{tikz}
\usepackage[outline]{contour}
\usepackage{multicol}
\usetikzlibrary{tikzmark}
\usetikzlibrary{shapes.symbols, positioning} % Load necessary libraries
\usetikzlibrary{fit}
% https://nanx.me/wordcloud/ generates with font sizes
% 100 s1
% 90 s2
% 79 s3
% 65 s4
% 44 s5
% 10 s6
% 0.577, 0.520, 0.456, 0.376, 0.254, 0.058
% 99.333, 89.786, 78.701, 64.788, 43.831, 10.000
% 69.6719, 62.8743, 55.1554, 45.4114, 30.6885, 7.0000
\tikzset{s1/.style={anchor=base,font={\fontsize{70pt}{12pt}\selectfont\bfseries}, headtext},
        s2/.style={anchor=base,font={\fontsize{63pt}{12pt}\selectfont}, color1!80},
        s3/.style={anchor=base,font={\fontsize{55pt}{12pt}\selectfont}, headTL},
        s4/.style={anchor=base,font={\fontsize{45pt}{12pt}\selectfont}, color2},
        s5/.style={anchor=base,font={\fontsize{30pt}{12pt}\selectfont}, color0!50},
%        s6/.style={font=\small, lightgray},
        s6/.style={anchor=base,font={\fontsize{7pt}{5pt}\selectfont}, color0},
        %every node/.style = {anchor=base} %, text depth=0pt, inner sep=+0pt, font=\vphantom{Ag}
}
\newcommand{\cloud}{
    \contourlength{1.5pt}
    \begin{center}\resizebox{0.7\textwidth}{!}{
        \begin{tikzpicture}
            %\node at (0, 0) {};
        \begin{scope}[scale=0.026458,yscale=-1]
            % take svg, and replace
            % <text text-anchor="middle" transform="translate\((.*)\),rotate\((.*)\)"\s+style="font-size: 10px; (.*)">\s*(.*)\s*</text>
            % to
            % \\node[s6] at ($1) {\\rotatebox{$2}{$4}};
            % in svg coords points to bottom anchor (# marked point below)
            % +-----+
            % | text|
            % +--#--+
            % Main text
%            \draw (196,-49) circle (5cm);
            \node[s1] at (196,-49) {\scalebox{0.89}[1]{\contour{color1}{Systems}}};
            \node[s2, rotate=0]  (dev) at (-152,11) {development};
            \node[s3, rotate=0]  (app) at (239,66) {Applications};
%            \draw (sys.center) -- (dev.center) (sys.base)   -- (dev.base);
%            \draw (sys.base) -- (0,0)
%            \foreach \o [count=\oi from 0] in {0.1, 0.15, 0.3, 0.5, 1} {%
%                \pgfmathtruncatemacro{\angle}{90 * \oi/4}
%                \node[s4, opacity=\o, rotate=\angle, rotate=45]  at (6,-93) {Linux};
%            }
            \node[s4, rotate=45] at (6,-93) {Linux};
            \node[s4, rotate=0]  at (80,115) {management};
            \node[s4, rotate=0]  at (222,3) {technical};
            \node[s5, rotate=45]  at (207,-141) {Design};
            \node[s5, rotate=45]  at (156,-145) {Creation};
            \node[s5, rotate=45]  at (-207,-173) {Security};
            \node[s5, rotate=45]  at (-278,110) {Website};
            \node[s5, rotate=45]  at (-187,103) {portals};
            \node[s5, rotate=45]  at (-91,-193) {architecture};
            \node[s5, rotate=0]  at (-31,42) {product};
            \node[s5, rotate=45]  at (-155,124) {functional};
            \node[s5, rotate=0]  at (-245,46) {automated};
            \node[s5, rotate=45]  at (-235,116) {code};
            \node[s5, rotate=0]  at (325,-99) {solutions};
            \node[s5, rotate=0]  at (-57,154) {developers};
            \node[s5, rotate=45]  at (136,158) {team};
            \node[s5, rotate=45]  at (111,-162) {quality};
            \node[s5, rotate=45]  at (105,-207) {writing};
            \node[s6, rotate=0]  at (22,-76) {Expertise};
            \node[s6, rotate=0]  at (-72,80) {Formulating};
            \node[s6, rotate=45]  at (-103,-59) {Software};
            \node[s6, rotate=45]  at (-100,-132) {Windows};
            \node[s6, rotate=0]  at (116,138) {Qnx};
            \node[s6, rotate=0]  at (244,50) {IoT};
            \node[s6, rotate=45] at (-167,-49) {Environments};
            \node[s6, rotate=45]  at (-123,-59) {DatabaseDriven};
            \node[s6, rotate=0]  at (-47,52) {Enhanced};
            \node[s6, rotate=45]  at (-131,-137) {SQL};
            \node[s6, rotate=45]  at (162,18) {Query};
            \node[s6, rotate=45]  at (29,-165) {Administration};
            \node[s5, rotate=45]  at (298,-179) {Performance};
            \node[s6, rotate=45]  at (-202,61) {docker};
            \node[s6, rotate=45]  at (-74,126) {ssh};
            \node[s6, rotate=45]  at (163,-35) {Tuning};
            \node[s5, rotate=45]  at (37,-192) {requirements};
            \node[s4, rotate=0]  at (-223,-88) {Optimization};
        \end{scope}\end{tikzpicture}
    }\end{center}
}
\newcommand{\blindtext}{The lipsum package is for producing placeholder text. It is useful for examples, in which you need to show some text to reproduce something, but the text itself doesn't matter. For example if somebody asks about a problem with page headers on the second page of their article, their example document needs to be at least two pages long. That's easy with the lipsum package.
Alternatively you could actually type random text in your code, but this makes it harder to see the code which actually matters.}

\newcommand{\headercols}[3]{
    \setlength{\tabcolsep}{3pt}
    \begin{tabular}{ p{0.333\textwidth} p{0.333\textwidth} p{0.333\textwidth} }
        #1 & %
        #2 & %
        #3 \\
    \end{tabular}
}


%\addtolength{\parskip}{-5pt}
\AtBeginDocument{\recomputelengths}
